\section{Background and Motivation}\label{sec:bg}
Dynamic optimization problems, such as finding the shortest path between two locations, can be performed on graphs with nodes and edges. However, this is traditionally confined to a single optimization constraint (or objective).
In many real-world applications, decision-making involves the consideration of multiple, often competing, criteria. To solve these problems, multi-objective optimization (MO) aims to simultaneously minimize (or maximize) a vector of objective functions.
Generally, there is not a single solution that minimizes all objectives, especially when dealing with objective functions that lead to non-convex optimization (e.g., weather).
% non-convex objective functions (e.g. weather).
Instead, MO finds the non-dominated, or \textit{Pareto-optimal} solutions, such that no solution improves one objective without worsening at least one other.
Formally, this can be stated as follows for $d$ objectives:
\begin{equation}
    \min_{x\in X} \left[ f_1(x),\cdots,f_d(x)  \right]
\end{equation}
where $X\subseteq\mathbb{R}^n$ represents the feasible solution space (e.g. all potential paths in a graph), and $f_i(x)$ represents the objective function for the $i$-th objective. 

In a graph setting, this allows multiple feasible Pareto-optimal solutions on each edge/node. This requires the use of labels to track the candidate solutions, where a label $l=(v,\hat{c})$ represents a specific node $v$ and a given feasible solution cost vector $\hat{c}$ of length equal to the number of objectives $d$. The number of feasible, Pareto-optimal solutions is unbounded, and each potential new path to a node can introduce a new feasible solution.  This results in an exponential growth in Pareto-optimal solution labels with the number of nodes and objectives in the graph.
This uniquely differentiates MO from existing graph algorithms that traditionally bound their time and space complexity by a polynomial function of the number of nodes and edges. 
% Unlike traditional graph algorithms, where irregular nodes and edges dictate complexity, MO complexity is a function of the number of labels, dictated by the number of nodes and objectives.
 % Unlike traditional graph algorithms with variability based on the number of nodes and edges, this introduces a third dimension of computational and storage variability.

Due to the exponential time and space complexity of managing solution labels in multi-objective graph optimization, the MO algorithms do not directly map to traditional graph frameworks. A systematic methodology for multi-objective graph problems is needed to solve this challenge. First, graphs need to be extended with multiple attributes (costs), adding additional metadata to traditional graph data structures with nodes, edges, and edge costs. 
For multi-objective graph problems such as multi-objective shortest path (MOSP), solving for the full set of Pareto-optimal solutions is complex and has been proven to be NP-hard, with complexity increasing exponentially with the number of nodes in the graph and with the number of objectives. These problems often require state space reduction of the infeasible regions in the graph, and additional metadata, such as heuristics, to aid the multi-objective optimizer in computing the Pareto-optimal solution front and reducing computational complexity.

In a recent survey paper, a lack of benchmarks and datasets for multi-objective graph optimization was identified as a key challenge for MO. Prior research used road networks with limited objectives or synthetic graphs with random objective functions. The survey paper also identified existing MO algorithms; however, each algorithm has unique input/output data structures, with no unifying framework for multi-objective graph optimization. Therefore, we propose the creation of a framework to address these limitations. 

% Our objective is to create a framework where one can create graphs for multi-objective optimization and explore existing multi-objective algorithms. Therefor, there is a need to... 
We aim to create a framework to 1) generate graphs for multi-objective optimization, and 2) explore existing multi-objective algorithms. The proposed framework will now be discussed in more detail, focusing on real-world applications.

% Our framework needs to 1) generate multi-attribute graphs (MAGs), which requires constructing the static graph (vertices and edges) and the computation of edge costs for each objective function, and 2) support running an MO algorithm, which may require computing additional metadata (such as heuristics) before performing the MO operation.  
% The proposed framework will now be discussed in more detail, focusing on real-world applications. 




% Our framework needs to
% \begin{enumerate}
%     \item generate \textbf{Multi-Attribute Graphs (MAGs)} on which to 
%     \item support running any \textbf{Multi-objective Optimization (MO)} algorithm. 
% \end{enumerate}

% Generating MAGs requires two key components: the static graph construction (vertices and edges), and the computation of multiple sets of edge costs, one for each objective function. 
% Once these graphs are constructed, many MO algorithms then require additional metadata, such as heuristics, to be generated before finally performing the multi-objective optimization.
% Depending on the objectives under consideration, this may require state-expanding the graph to include additional metadata (e.g., adding temporal state to an existing spatial graph for proper weather forecasting). 
% Our objective is to apply these concepts to real-world applications: generating MAGs and performing multi-objective shortest path (MOSP), a representative MO algorithm, as a challenge problem. The proposed framework will now be discussed in more detail. 

% Graph problems requires require the formulation of labels to compute the pareto-optimal solutions. More logic on why exponential growth with num of nodes. end with NP-hard.
% use feasible solutions instead of paths
% maybe add prior work- reducing complexity at lower objectives but unknown if they work at higher 

% not traditional graph or traditional graph algirhtm. Cannot take existing graphs or frameworks.
% Need multiple attributes on the graph
% Need more metadata such as heursitics that helps the multi-objective optimizer cmpute the pareto-front and reduce the computational complexity
% Then complexity argument  

% Then summarize lack of graphs, and lack of frameworks. 
%Goals:
% - Generate MAG
% - Support any multi-objective optimization algorithms
% Paper focusing on generating MAGs and performing representative MO algirhtms (MOSP) as a challenge problem


% Single-objective graph optimization problems such as single-objective shortest path can be computed using dynamic programming; however, expanding to multiple objectives requires the following:
% \begin{enumerate}
%     \item \textbf{Multi-Attribute Graphs (MAGs)} on which to perform the multi-objective optimization. This requires the computation of multiple objective costs for each edge. Depending on the objectives under consideration, this may require state-expanding the graph to include additional metadata (e.g., adding temporal state to an existing spatial graph for proper weather forecasting). 

%     \item A framework to perform the \textbf{Multi-objective Optimization (MO)} algorithms on MAGs. Heuristics may be needed to guide towards convergence and reduce the runtime complexity of these algorithms. Admissible heuristics (algorithm dependent) can improve performance with no loss in accuracy.
% \end{enumerate}

% remove specifics, talk instead about requirements for framework
% lack of benchmarks and datasets identified in recent survey paper. one objectives of this work is to address that limitation. 
% - - prior research uses road networks with a limited number of objectives
% existing MO algorithms, and building a framework to perform these algorithms
% However, there is a lack of real-world MAGs, with prior research using synthetic graphs with random costs or DIMACS road networks with limited objectives and results. We also note an absence of frameworks for multi-objective optimization on MAGs. Our goal is to build a framework for generating MAGs and feeding these standardized inputs to MO algorithms. To evaluate this framework, we choose Tool for Multi-objective Planning and Asset Routing (TMPLAR) as a case study for this paper, generating real-world MAGs for naval vessel navigation and performing MOSP to find the complete set of Pareto-optimal paths between the source and destination. Two different representative MOSP algorithms will be chosen for this paper (NAMOA* and EMOA*), with the ability to expand to any multi-objective optimization problem. 
% The framework will now be explored in more detail. 





























% , and a framework that can support performing any MO algorithm on MAGs. Our goal is to introduce a framework capable of exploring multi-objective optimization on multi-attribute graphs.
% First, MAGs suitable for these types of optimization problems need to be generated from real-world datasets. To demonstrate, we will build MAGs for naval vessel navigation using the Tool for Multi-objective Planning and Asset Routing (TMPLAR). Second, we will use MOSP as our challenge problem to show the efficacy of the framework. Two different representative MOSP algorithms will be chosen for this paper (NAMOA* and EMOA*), with the ability to expand to any multi-objective optimization problem. The framework will now be explored in more detail. 



% Dynaimc optimization on a graph. Shortest path can be done on a graph setting where you have nodes, edges, and the goal is to find a shortest path between two nodes. However, if we want to consider multiple objectives, we need to add the following to the graph: 
% whose goal is to find the Pareto-optimal front. Goal is to find multiple solutions the are all optimal
% Get into why the problem is so complex. MO is exponential in num nodes in graph -> very hard problem.
% can perform DP on single-objective, however to expand to multi-objective...
% TO evaluate this, we need the following:
% - Graph with multiple attributes to do MO
% - (1) required: compute multiple objective costs
%  - - may also require state-space expansion 
% - (2) run MO optimization (want to support any algorithms)
% - - may need heuristics

% Lack of graphs and frameworks to do MO on these types of MAGs.
% Our goal is to generate MAGs suitable from these MO
% To demonstrate, we have picked MOSP as our challenge problem to show the efficacy of the framework. Two different representative algorithms will be chosen for this paper


% In many real-world applications, decision-making involves consideration of multi-state systems (e.g. systems with spatial and temporal elements). For example, solving shortest-path problems with motion and time constraints necessitates using a multi-state system considering the spatial location and the temporal time. With these spatio-temporal states, the calculation of the shortest path can be reduced to a forward or backward dynamic programming problem. The forward dynamic programming equation for a single objective shortest path with spatio-temportal constraints can be written as
% \begin{equation}
%     J()
% \end{equation}
% where 

% However, this is constrained to optimizing a single cost, or objective. In many dynamic situations, multiple, often competing criteria need to be considered. To solve these problems, multi-objective optimization (MO) aims to simultaneously minimize (or maximize) a vector of objective functions. Formally, this can be stated as follows for $d$ objectives:
% \begin{equation}
%     \min_{x\in X} \left[ f_1(x),\cdots,f_d(x)  \right]
% \end{equation}
% where $X\subseteq\mathbb{R}^n$ represents the feasible solution space, and $f_i(x)$ represents the objective function for the $i$-th objective. Generally, there is not a single solution that minimizes all objectives, especially when dealing with non-convex objective data (e.g. weather). Instead, MO finds the non-dominated, or \textit{Pareto-optimal} solutions, such that no solution improves one objective without worsening at least one other.


% - Multi-objective optimization
% - - Why we need multiple objectives - non-convex, weather, etc
% - Introduce DP problem
% - Lack of MAGs
% - Need for a unified framework for MAG generation and MO 
% In many real-world applications, decision-making involves the consideration of multiple, often competing, criteria. To solve these problems, multi-objective optimization (MO) aims to simultaneously minimize (or maximize) a vector of objective functions. Formally, this can be stated as follows for $d$ objectives:
% \begin{equation}
%     \min_{x\in X} \left[ f_1(x),\cdots,f_d(x)  \right]
% \end{equation}
% where $X\subseteq\mathbb{R}^n$ represents the feasible solution space, and $f_i(x)$ represents the objective function for the $i$-th objective. Generally, there is not a single solution that minimizes all objectives, especially when dealing with non-convex objective data (e.g. weather). Instead, MO finds the non-dominated, or \textit{Pareto-optimal} solutions, such that no solution improves one objective without worsening at least one other.
% Applying MO to graph problems necessitates the use of multi-attribute graphs (MAGs) that combine a standard graph with multiple attributes, or edge costs. Each edge on the graph has a vector of costs with length equal to the number of objectives. 

% To incorporate objectives based on time-varying data, such as traffic congestion or weather, spatial graphs need to be augmented with temporal data, creating spatio-temporal graphs. However, expanding existing spatial graphs into a spatio-temporal setting can increase the size of the graphs by an order of magnitude. For complex MO algorithms, this can be costly. This necessitates the use of state-space reduction on the spatio-temporal MAGs to constrain the explosion of computational complexity. 

% Existing graph MO algorithms perform a set of operators on MAGs to produce Pareto-optimal solutions. Many MO problems, such as multi-objective shortest path (MOSP), have been proven to be NP-hard. They rely on a set of consistent heuristics (e.g. cost-to-goal for MOSP) to improve runtime complexity and converge on the set of Pareto-optimal solutions. With a single-objective graph $G(V,E)$, the graph varies only with the number of vertices $|V|$ and the number of edges $|E|$. Thus, the space overhead can be calculated as a function of $|V|$ and $|E|$. However, a $MAG(V,E,\hat{c}$) adds a third dimension: the $d$ objective costs. Unlike single-objective graph problems, with a single value per edge or vertex, with multiple objectives many candidate costs can exist on each node. This necessitates the use of a Label$(n, \hat{c})$, a tuple of a node and its associated cost vector of length $d$ objectives, to describe a unit of work in MO algorithms. Since an indeterminate number labels can exist on each vertex, the algorithms cannot preallocate memory based on the number of nodes and edges, and must instead opt for dynamic memory allocation. This variation in number of labels introduces both space and computational irregularly, making MO very different from traditional graph algorithms. 

% THis is hard because as you increase nodes, the problem grows exponentially with the number of nodes (2^n)







% Solve the problem, and set up the spatio-teporal graphs with state space description. Time-varying objectives, so we need to do time-carying
% we need to track multi-objective data so need cost vectors
% --> we now have graph with space, time, and obj to support MOO
% We call this a label, and we will perform a method to determine what solutions are feasible
% dont no how many solns are there, so num of labels can grow exponentially
% NP hard
% so we need to support this formulation
% lack of graphs, lack of objectives
%  ue road networks, but show a real-world opt in TMPLAR (makes a mag and performs MOO using SOTA algs)