\documentclass[10pt,letterpaper,compsoc,conference]{iiswc25}

%% INCLUDED PACKAGES: DO NOT REMOVE ANY OF THESE
\usepackage{cite}
\usepackage{amsmath,amssymb,amsfonts}
%\usepackage{algorithmic}
\usepackage{algorithm}
\usepackage[noend]{algpseudocode}
\renewcommand{\algorithmicrequire}{\textbf{Input:}}
\renewcommand{\algorithmicensure}{\textbf{Output:}}
% \newcommand{\blu}[1]{\textcolor{blue}{#1}}
% \newcommand{\red}[1]{\textcolor{red}{#1}}
% \newcommand{\blx}[1]{\textcolor{blue}{\sout{#1}}}

\usepackage{graphicx}
\usepackage[dvipsnames]{xcolor}
\usepackage[final]{microtype}
\usepackage[italic]{mathastext}
\usepackage{libertine}
\usepackage[T1]{fontenc}
\usepackage{textcomp}
\usepackage[varqu,varl]{zi4}
\usepackage[all]{nowidow}
%\usepackage[auth-lg,affil-it]{authblk}
\usepackage[keeplastbox]{flushend}
\usepackage{fancyhdr}

\usepackage{graphicx}
\usepackage{textcomp}
\usepackage{xcolor}
\usepackage{multirow}
\usepackage{multicol}
\usepackage{tabularx}
%\usepackage{adjustbox}
\usepackage{makecell}
\usepackage{subcaption}
\usepackage{setspace}
\usepackage{printlen}
\newcommand{\red}[1]{\textcolor{red}{#1}}
\newcommand{\blue}[1]{\textcolor{blue}{#1}}


%% ADD YOUR OTHER PACKAGES HERE
\usepackage[labelfont=bf, textfont=normalfont]{caption}

%% ADD YOUR OTHER PACKAGES ABOVE THIS LINE

%%%%%%%%%%% ---SETME-----%%%%%%%%%%%%%
\newcommand{\iiswcsubmissionnumber}{XX}
%%%%%%%%%%%%%%%%%%%%%%%%%%%%%%%%%%%%

\fancypagestyle{firstpage}{
  \fancyhf{}
  \renewcommand{\headrulewidth}{0pt}
  \fancyhead[C]{\vspace{10pt}\normalsize{2025 Submission
      \textbf{\#\iiswcsubmissionnumber} -- Confidential Draft -- Do
      NOT Distribute!!} \\\vspace{-25pt}} 
  \fancyfoot[C]{\thepage}
}

\begin{document}


%% EDIT TITLE BELOW

\title{MAGMO: A Framework for Multi-objective Graph Optimizations}


%% DO NOT EDIT THE FOLLOWING

%\renewcommand\Authsep{\qquad}
%\renewcommand\Authand{\qquad}
%\renewcommand\Authands{\qquad}


%% EDIT AUTHOR LIST BELOW

%\author{Author1 Name}
%\author{Author2 Name}
%\author{Author3 Name}
%\affil{Full Name of Awesome School}



%%% ALTERNATIVE FORMAT FOR MULTIPLE SCHOOLS:
%%% 
% \author[1]{Author1 Name}
% \author[2]{Author2 Name}
% \author[2]{Author3 Name}
% \author[1]{Author4 Name}
% \affil[1]{Full Name of Awesome School}
% \affil[2]{Full Name of Awesomer School}



\maketitle
\thispagestyle{firstpage}
\pagestyle{plain}

%% EDIT YOUR PAPER'S CONTENTS BELOW


\begin{abstract}
Dynamic graph optimization problems are traditionally confined to a single optimization constraint. However, many real-world decision-making applications require considering multiple, often competing, objectives. Multi-objective optimization algorithms have been found to have exponential time and space complexity, preventing the use of existing graph frameworks. Literature lacks a systemic methodology for the computational acceleration of multi-objective optimization in graph settings. Such a framework should enable the efficient generation of multi-attribute graphs (MAGs) from real-world datasets and apply hardware acceleration to multi-objective optimization (MO), laying the groundwork for real-time decision-making applications. 
This paper presents MAGMO, a novel framework for multi-objective graph optimizations, and makes the following contributions: 1) A full end-to-end multi-objective graph framework that takes real-world applications from marine (TMPLAR) and road network (DIMACS) applications, using real input data to build MAGs, and applying MO to generate cost-unique Pareto-optimal solutions. 2) With a focus on multi-objective shortest path (MOSP) as a challenge problem, the characterization and comparison of four sequential algorithms are explored, identifying this NP-hard MO problem as the key performance bottleneck in MAGMO. 3) Sequential characterization exposes the key challenges for MOSP acceleration center around operations on unstructured graph data. 4) Prior work has explored parallelism in MOSP; however, based on the insights from MOSP characterization, novel parallel algorithms are proposed, showing the performance calling potential on modern parallel machines. An experimental evaluation using the NVIDIA GH200 Superchip’s 72-core Arm-based CPU demonstrates the performance scaling potential of the MAGMO framework as a system that enables end-to-end multi-objective optimization in real-world applications.
\end{abstract}




\input{sections/intro}
\section{Background and Motivation}\label{sec:bg}
Dynamic optimization problems, such as finding the shortest path between two locations, can be performed on graphs with nodes and edges. However, this is traditionally confined to a single optimization constraint (or objective).
In many real-world applications, decision-making involves the consideration of multiple, often competing, criteria. To solve these problems, multi-objective optimization (MO) aims to simultaneously minimize (or maximize) a vector of objective functions.
Generally, there is not a single solution that minimizes all objectives, especially when dealing with objective functions that lead to non-convex optimization (e.g., weather).
% non-convex objective functions (e.g. weather).
Instead, MO finds the non-dominated, or \textit{Pareto-optimal} solutions, such that no solution improves one objective without worsening at least one other.
Formally, this can be stated as follows for $d$ objectives:
\begin{equation}
    \min_{x\in X} \left[ f_1(x),\cdots,f_d(x)  \right]
\end{equation}
where $X\subseteq\mathbb{R}^n$ represents the feasible solution space (e.g. all potential paths in a graph), and $f_i(x)$ represents the objective function for the $i$-th objective. 

In a graph setting, this allows multiple feasible Pareto-optimal solutions on each edge/node. This requires the use of labels to track the candidate solutions, where a label $l=(v,\hat{c})$ represents a specific node $v$ and a given feasible solution cost vector $\hat{c}$ of length equal to the number of objectives $d$. The number of feasible, Pareto-optimal solutions is unbounded, and each potential new path to a node can introduce a new feasible solution.  This results in an exponential growth in Pareto-optimal solution labels with the number of nodes and objectives in the graph.
This uniquely differentiates MO from existing graph algorithms that traditionally bound their time and space complexity by a polynomial function of the number of nodes and edges. 
% Unlike traditional graph algorithms, where irregular nodes and edges dictate complexity, MO complexity is a function of the number of labels, dictated by the number of nodes and objectives.
 % Unlike traditional graph algorithms with variability based on the number of nodes and edges, this introduces a third dimension of computational and storage variability.

Due to the exponential time and space complexity of managing solution labels in multi-objective graph optimization, the MO algorithms do not directly map to traditional graph frameworks. A systematic methodology for multi-objective graph problems is needed to solve this challenge. First, graphs need to be extended with multiple attributes (costs), adding additional metadata to traditional graph data structures with nodes, edges, and edge costs. 
For multi-objective graph problems such as multi-objective shortest path (MOSP), solving for the full set of Pareto-optimal solutions is complex and has been proven to be NP-hard, with complexity increasing exponentially with the number of nodes in the graph and with the number of objectives. These problems often require state space reduction of the infeasible regions in the graph, and additional metadata, such as heuristics, to aid the multi-objective optimizer in computing the Pareto-optimal solution front and reducing computational complexity.

In a recent survey paper, a lack of benchmarks and datasets for multi-objective graph optimization was identified as a key challenge for MO. Prior research used road networks with limited objectives or synthetic graphs with random objective functions. The survey paper also identified existing MO algorithms; however, each algorithm has unique input/output data structures, with no unifying framework for multi-objective graph optimization. Therefore, we propose the creation of a framework to address these limitations. 

% Our objective is to create a framework where one can create graphs for multi-objective optimization and explore existing multi-objective algorithms. Therefor, there is a need to... 
We aim to create a framework to 1) generate graphs for multi-objective optimization, and 2) explore existing multi-objective algorithms. The proposed framework will now be discussed in more detail, focusing on real-world applications.

% Our framework needs to 1) generate multi-attribute graphs (MAGs), which requires constructing the static graph (vertices and edges) and the computation of edge costs for each objective function, and 2) support running an MO algorithm, which may require computing additional metadata (such as heuristics) before performing the MO operation.  
% The proposed framework will now be discussed in more detail, focusing on real-world applications. 




% Our framework needs to
% \begin{enumerate}
%     \item generate \textbf{Multi-Attribute Graphs (MAGs)} on which to 
%     \item support running any \textbf{Multi-objective Optimization (MO)} algorithm. 
% \end{enumerate}

% Generating MAGs requires two key components: the static graph construction (vertices and edges), and the computation of multiple sets of edge costs, one for each objective function. 
% Once these graphs are constructed, many MO algorithms then require additional metadata, such as heuristics, to be generated before finally performing the multi-objective optimization.
% Depending on the objectives under consideration, this may require state-expanding the graph to include additional metadata (e.g., adding temporal state to an existing spatial graph for proper weather forecasting). 
% Our objective is to apply these concepts to real-world applications: generating MAGs and performing multi-objective shortest path (MOSP), a representative MO algorithm, as a challenge problem. The proposed framework will now be discussed in more detail. 

% Graph problems requires require the formulation of labels to compute the pareto-optimal solutions. More logic on why exponential growth with num of nodes. end with NP-hard.
% use feasible solutions instead of paths
% maybe add prior work- reducing complexity at lower objectives but unknown if they work at higher 

% not traditional graph or traditional graph algirhtm. Cannot take existing graphs or frameworks.
% Need multiple attributes on the graph
% Need more metadata such as heursitics that helps the multi-objective optimizer cmpute the pareto-front and reduce the computational complexity
% Then complexity argument  

% Then summarize lack of graphs, and lack of frameworks. 
%Goals:
% - Generate MAG
% - Support any multi-objective optimization algorithms
% Paper focusing on generating MAGs and performing representative MO algirhtms (MOSP) as a challenge problem


% Single-objective graph optimization problems such as single-objective shortest path can be computed using dynamic programming; however, expanding to multiple objectives requires the following:
% \begin{enumerate}
%     \item \textbf{Multi-Attribute Graphs (MAGs)} on which to perform the multi-objective optimization. This requires the computation of multiple objective costs for each edge. Depending on the objectives under consideration, this may require state-expanding the graph to include additional metadata (e.g., adding temporal state to an existing spatial graph for proper weather forecasting). 

%     \item A framework to perform the \textbf{Multi-objective Optimization (MO)} algorithms on MAGs. Heuristics may be needed to guide towards convergence and reduce the runtime complexity of these algorithms. Admissible heuristics (algorithm dependent) can improve performance with no loss in accuracy.
% \end{enumerate}

% remove specifics, talk instead about requirements for framework
% lack of benchmarks and datasets identified in recent survey paper. one objectives of this work is to address that limitation. 
% - - prior research uses road networks with a limited number of objectives
% existing MO algorithms, and building a framework to perform these algorithms
% However, there is a lack of real-world MAGs, with prior research using synthetic graphs with random costs or DIMACS road networks with limited objectives and results. We also note an absence of frameworks for multi-objective optimization on MAGs. Our goal is to build a framework for generating MAGs and feeding these standardized inputs to MO algorithms. To evaluate this framework, we choose Tool for Multi-objective Planning and Asset Routing (TMPLAR) as a case study for this paper, generating real-world MAGs for naval vessel navigation and performing MOSP to find the complete set of Pareto-optimal paths between the source and destination. Two different representative MOSP algorithms will be chosen for this paper (NAMOA* and EMOA*), with the ability to expand to any multi-objective optimization problem. 
% The framework will now be explored in more detail. 





























% , and a framework that can support performing any MO algorithm on MAGs. Our goal is to introduce a framework capable of exploring multi-objective optimization on multi-attribute graphs.
% First, MAGs suitable for these types of optimization problems need to be generated from real-world datasets. To demonstrate, we will build MAGs for naval vessel navigation using the Tool for Multi-objective Planning and Asset Routing (TMPLAR). Second, we will use MOSP as our challenge problem to show the efficacy of the framework. Two different representative MOSP algorithms will be chosen for this paper (NAMOA* and EMOA*), with the ability to expand to any multi-objective optimization problem. The framework will now be explored in more detail. 



% Dynaimc optimization on a graph. Shortest path can be done on a graph setting where you have nodes, edges, and the goal is to find a shortest path between two nodes. However, if we want to consider multiple objectives, we need to add the following to the graph: 
% whose goal is to find the Pareto-optimal front. Goal is to find multiple solutions the are all optimal
% Get into why the problem is so complex. MO is exponential in num nodes in graph -> very hard problem.
% can perform DP on single-objective, however to expand to multi-objective...
% TO evaluate this, we need the following:
% - Graph with multiple attributes to do MO
% - (1) required: compute multiple objective costs
%  - - may also require state-space expansion 
% - (2) run MO optimization (want to support any algorithms)
% - - may need heuristics

% Lack of graphs and frameworks to do MO on these types of MAGs.
% Our goal is to generate MAGs suitable from these MO
% To demonstrate, we have picked MOSP as our challenge problem to show the efficacy of the framework. Two different representative algorithms will be chosen for this paper


% In many real-world applications, decision-making involves consideration of multi-state systems (e.g. systems with spatial and temporal elements). For example, solving shortest-path problems with motion and time constraints necessitates using a multi-state system considering the spatial location and the temporal time. With these spatio-temporal states, the calculation of the shortest path can be reduced to a forward or backward dynamic programming problem. The forward dynamic programming equation for a single objective shortest path with spatio-temportal constraints can be written as
% \begin{equation}
%     J()
% \end{equation}
% where 

% However, this is constrained to optimizing a single cost, or objective. In many dynamic situations, multiple, often competing criteria need to be considered. To solve these problems, multi-objective optimization (MO) aims to simultaneously minimize (or maximize) a vector of objective functions. Formally, this can be stated as follows for $d$ objectives:
% \begin{equation}
%     \min_{x\in X} \left[ f_1(x),\cdots,f_d(x)  \right]
% \end{equation}
% where $X\subseteq\mathbb{R}^n$ represents the feasible solution space, and $f_i(x)$ represents the objective function for the $i$-th objective. Generally, there is not a single solution that minimizes all objectives, especially when dealing with non-convex objective data (e.g. weather). Instead, MO finds the non-dominated, or \textit{Pareto-optimal} solutions, such that no solution improves one objective without worsening at least one other.


% - Multi-objective optimization
% - - Why we need multiple objectives - non-convex, weather, etc
% - Introduce DP problem
% - Lack of MAGs
% - Need for a unified framework for MAG generation and MO 
% In many real-world applications, decision-making involves the consideration of multiple, often competing, criteria. To solve these problems, multi-objective optimization (MO) aims to simultaneously minimize (or maximize) a vector of objective functions. Formally, this can be stated as follows for $d$ objectives:
% \begin{equation}
%     \min_{x\in X} \left[ f_1(x),\cdots,f_d(x)  \right]
% \end{equation}
% where $X\subseteq\mathbb{R}^n$ represents the feasible solution space, and $f_i(x)$ represents the objective function for the $i$-th objective. Generally, there is not a single solution that minimizes all objectives, especially when dealing with non-convex objective data (e.g. weather). Instead, MO finds the non-dominated, or \textit{Pareto-optimal} solutions, such that no solution improves one objective without worsening at least one other.
% Applying MO to graph problems necessitates the use of multi-attribute graphs (MAGs) that combine a standard graph with multiple attributes, or edge costs. Each edge on the graph has a vector of costs with length equal to the number of objectives. 

% To incorporate objectives based on time-varying data, such as traffic congestion or weather, spatial graphs need to be augmented with temporal data, creating spatio-temporal graphs. However, expanding existing spatial graphs into a spatio-temporal setting can increase the size of the graphs by an order of magnitude. For complex MO algorithms, this can be costly. This necessitates the use of state-space reduction on the spatio-temporal MAGs to constrain the explosion of computational complexity. 

% Existing graph MO algorithms perform a set of operators on MAGs to produce Pareto-optimal solutions. Many MO problems, such as multi-objective shortest path (MOSP), have been proven to be NP-hard. They rely on a set of consistent heuristics (e.g. cost-to-goal for MOSP) to improve runtime complexity and converge on the set of Pareto-optimal solutions. With a single-objective graph $G(V,E)$, the graph varies only with the number of vertices $|V|$ and the number of edges $|E|$. Thus, the space overhead can be calculated as a function of $|V|$ and $|E|$. However, a $MAG(V,E,\hat{c}$) adds a third dimension: the $d$ objective costs. Unlike single-objective graph problems, with a single value per edge or vertex, with multiple objectives many candidate costs can exist on each node. This necessitates the use of a Label$(n, \hat{c})$, a tuple of a node and its associated cost vector of length $d$ objectives, to describe a unit of work in MO algorithms. Since an indeterminate number labels can exist on each vertex, the algorithms cannot preallocate memory based on the number of nodes and edges, and must instead opt for dynamic memory allocation. This variation in number of labels introduces both space and computational irregularly, making MO very different from traditional graph algorithms. 

% THis is hard because as you increase nodes, the problem grows exponentially with the number of nodes (2^n)







% Solve the problem, and set up the spatio-teporal graphs with state space description. Time-varying objectives, so we need to do time-carying
% we need to track multi-objective data so need cost vectors
% --> we now have graph with space, time, and obj to support MOO
% We call this a label, and we will perform a method to determine what solutions are feasible
% dont no how many solns are there, so num of labels can grow exponentially
% NP hard
% so we need to support this formulation
% lack of graphs, lack of objectives
%  ue road networks, but show a real-world opt in TMPLAR (makes a mag and performs MOO using SOTA algs)
\section{MAGMO Framework}

\begin{figure}
    \centering
    \includegraphics[width=\linewidth]{figs/ECHO.png}
    \caption{MAGMO Dataflow}
    \label{fig:magmo}
\end{figure}

% ToMAtO: Tool for Multi-Attribute-graph Optimization

% ToMAGO: Tool for Multi-Attribute Graph Optimization

% ToMOGO: Tool for Multi-Objective Graph Optimization

% MAGOT: Multi-Attribute Graph Optimization Tool

Prior works on multi-objective graph optimization struggle to find real-world graphs with high numbers of objectives. Additionally, open-source MO algorithms have distinct input data structures, making it difficult to compare competing solutions. We propose MAGMO, a framework for performing multi-objective optimization (MO) on generated multi-attribute graphs (MAGs), to address the challenges facing multi-objective graph optimization. As shown in Figure~\ref{fig:magmo}, MAGMO extracts a graph from real-world datasets, computes the multi-attribute edge costs, generates any required heuristics, and feeds these inputs into a multi-objective optimization algorithm. Each stage is optional and modified depending on the workload. The following discussion shows how an existing multi-objective application, the Tool for Multi-objective Planning and Asset Routing (TMPLAR), is mapped on MAGMO to perform naval vessel navigation on real-world land-sea graphs.

\subsection{Multi-Attribute Graphs (MAG)}
\begin{figure*}
        \centering
    \includegraphics[width=0.9\linewidth]{figs/Graph Transition.png}
    \caption{Graph Generation Phase}
    \label{fig:ge}
\end{figure*}
% want to support non-convex inputs since these are often needed in real-world applications. In addition to spatial information need temporal information to take into account time-evolving data and do things that require dynamic information for multi-objective optimization problems. 
% need to expand with spatial and temporal information. but this grows the number of nodes, which gives exponential growth in complexity, so we need to reduce this graph. to further understand this, take a look at motion planning
% motion planning as a case study to explore how this can be done
MAGMO introduces a standard format for MAGs. The graphs are stored in a compressed format for the input graph (usually compressed sparse row (CSR)), with a separate value array for each objective. MAGMO loads existing MAGs in this format and feeds them into the subsequent stages of the pipeline. For new graphs, MAGMO exposes two phases for graph construction: a graph extraction phase to extract a usable graph from input datasets, and an edge-cost computation phase to compute the edge costs for each objective. For graphs with preset edge costs (such as those in the DIMACS dataset [ref]), the graph and its edge costs are extracted in a single phase, and the edge cost computation is skipped. These two phases are explored in more detail.

\subsubsection{Graph Extraction}

Many real-world applications consider objective functions that form a non-convex optimization space. Such objective inputs require information not only on the physical locations but also on time. Expanding spatial graphs with temporal information allows for the use of time-varying data and more complex dynamic optimization. However, the temporal state expansion leads to an order-of-magnitude increase in the number of nodes. We know from Section~\ref{sec:bg} that the complexity of optimization increases exponentially with the number of nodes. Therefore, a reduction of the search space is performed after time expansion to allow tractable computation. A motion planning case study is presented to understand this process. 

Consider a naval vessel navigating through the ocean. A land/sea map is provided to show the valid navigable positions (sea) and the impassable locations (land). Using this map, each sea location becomes a node in the graph. Edges between these nodes are formed by connecting each node to its neighboring sea nodes. As shown in Figure~\ref{fig:ge}a, a constraint is given on the number of rings around the source node to search for neighbors. Increasing the number of ``hops'' allows for smoother navigation at the cost of increased connectivity (and therefore complexity). In the given example, one hop forms a ring with the eight immediate neighbors of a node, while two hops adds 16 more distant neighbors for a total of 24. Repeating this process for all valid nodes creates a spatial graph that is used in scenarios where routing is constrained to geographical feasibility alone. 

Temporal state metadata is added to the graph to enable realistic modeling of routing conditions and to account for time-dependent data such as vessel speed, weather, and operational schedules. The operator specifies a source node $v_s$ to estimate the time of arrival at each node $v_i$ in the graph. The minimum distance $\ell_i$ from $v_s$ to each node $v_i$ in the graph is found using a single-source shortest-path (SSSP) algorithm. With the naval navigation example, the Rhumb line distance between two nodes is used as the cost function for SSSP. Using $\ell_i$ and operator-specified vehicle speed constraints $V_{\text{min}}$ and $V_{\text{max}}$ (minimum and maximum velocity, respectively), the minimum ($T_{a,\text{min}}$) and maximum ($T_{a,\text{max}}$) time of arrival at each node $v_i$ is calculated. Using these time constraints, each node $v_i$ is expanded into multiple nodes representing discrete time intervals between $T_{a,\text{min}}$ and $T_{a,\text{max}}$. This enables routing with variable speed through realistic travel conditions. However, the resulting graph can have an order of magnitude increase in the number of nodes, resulting in a larger search space with an exponential increase in complexity.

To address the resulting increase in complexity, the operator supplies an upper and lower bound on the time of arrival at the destination, $T_{f,\text{max}}$ and $T_{f,\text{min}}$, respectively. Using an additional SSSP with the source set to the destination $v_f$, the minimum distance from the destination to each node $\ell_{b,i}$ is calculated. This allows the question to be answered: when must the vessel leave a node to reach the destination within the arrival constraints? Using $\ell_{b,i}$, and the $T_f$ and $V$ constraints, the corresponding minimum and maximum departure times for each node is are calculated ($T_{d,\text{min}}$ and $T_{d,\text{max}}$ respectively). Overlapping the arrival and departure time windows creates the earliest and latest times that the vessel can arrive at every node such that the vessel is still able to arrive at the destination, subject to the $T_f$ constraints. For a given node $v_i$, the time bounds are given by 
\begin{equation}
    T_{\text{min}}=\max(T_{a,\text{min}},T_{d,\text{min}})
\end{equation}
\begin{equation}
    T_{\text{max}}=\min(T_{a,\text{max}},T_{d,\text{max}})
\end{equation}
such that if $T_{min}\leq T_{max}$ then it is feasible to arrive at a node and depart in time to reach the destination. However, if $T_{min} > T_{max}$, then the node cannot be feasibly reached during the path planning, and thus pruned from the graph (Figure~\ref{fig:ge}b. This reduction in the search space (and number of nodes) leads to an improvement in the computational complexity of MO.

For the remaining nodes with valid time windows, each node is expanded into multiple nodes representing discrete times of arrival between $T_{\text{min}}$ and $T_{\text{max}}$ (Figure~\ref{fig:ge}c). The number of temporal nodes per geographic node is specified by the operator. Now, edges need to be formed between the time-expanded and state-spaced-reduced nodes. For a given node (with a given geographic and time state), all time-expanded nodes geographically adjacent (subject to the same ``hops'' constraint used in the spatial graph) are explored. The speed required to get to each adjacent node is calculated using the Rhumb line distance and the difference between the time states. If this meets the required speed constraints specified by the operator ($V_{\text{min}}$ and $V_{\text{max}}$), then the edge is added to the graph. Otherwise, the edge is not included. The resulting graph (Figure~\ref{fig:ge}d) reflects not only feasible states, but also valid methods of getting to each state. Additionally, by adding temporal data to the spatial graph, time-varying data can now be considered. 
%The computation of this multi-objective cost data will now be discussed.


\subsubsection{Edge Cost Computation}

% 1 - we can create a bunch of objectives
% 2- cannot be preprocessed into a single objective. Need to justify why we need a bunch of objectivess
A key component of MAG is multiple attributes or objectives on the graph. In MAGMO, this is represented by separate edge-cost value arrays for each objective that can be independently computed. MAGMO offers a modular interface for connecting different objective functions for different applications. For example, TMPLAR considers many vessel dynamics and weather objective functions to improve navigation. These include time, fuel, roll, pitch, vertical acceleration (V.ACCEL), horizontal acceleration (H.ACCEL), vertical bending moment (VBM), vertical shear force (VSF), wave height ($H_s$), wave period ($T_0$), and the relative sea direction ($\theta$). The general computational structure of TMPLAR objective costs is described next. 
%Each edge has a source node $v_s$ and a destination node $v_n$.

The time objective is calculated as the time delta ($\Delta t$) between the time state at the source node and the time state at the destination node. The wave height ($H_s$), wave period ($T_0$), and sea direction ($Dir$) are pulled from the latest ERA5 forecast data closest to the location and time state at each node. The relative sea direction ($\theta$) is calculated as the smallest angular distance between the vehicle's bearing and $Dir$, the sea direction. The Rhumb-line distance ($Dist$) is found between the two nodes, and the speed $s$ is then derived as the ratio $Dist/\Delta t$.

Multiple routing-related objectives are calculated using precomputed empirical datasets. These datasets store measured responses under varying operational and environmental conditions. Using five computed parameters ($Dist$, $s$, $H_s$, $T_0$, $\theta$), the closest match in the empirical dataset for an objective is derived using the Manhattan distance (L1-norm) method. For example, the fuel objective is precomputed by modeling the required propulsive power in calm water as in [37], as well as the power required by the added resistance of waves [38]. Optimizing fuel consumption involves tradeoffs with minimizing distance traveled and navigating areas of weather uncertainty. 
Therefore, it uses the speed, distance, and meteorology parameters for its computation.

The vessel's structural load objectives (V.ACCEL, H.ACCEL, VBM, and VSF) are crucial for minimizing wear and tear, thereby decreasing the amount of required maintenance and extending the vessel’s lifespan [39]. Moreover, the sea-keeping objectives (roll and pitch) reflect the vessel's ability to maintain stability and course in wave conditions. These objectives also use the five parameters ($Dist$, $s$, $H_s$, $T_0$, $\theta$) to look up precomputed empirical datasets derived from nonlinear wave-load analysis [40].

%Optimizing fuel consumption as an objective can results in trade-offs with minimizing distance traveled, especially in cases where the shortest distance path passes through an area of severe/hazardous weather. Often passing through such areas of high weather uncertainty increases overall fuel usage, while taking a diverting route around the severe weather zones often reduces fuel usage at the cost of increasing the overall distance of the journey. The ship response objectives are crucial for minimizing the ship’s wear-and-tear, thereby decreasing the amount of required maintenance and extending the vessel’s lifespan[39]. These parameters are calculated using a nonlinear wave-load analysis [40].

\subsection{Multi-Objective Graph Optimizations (MO)}

MAGMO explores the NP-hard multi-objective shortest path as a representative MO problem. Given a weighted graph with non-negative edge weights, the shortest path problem computes the minimum-cost path from a start node to a goal/destination node in the graph~\cite{tarjan_book}. In a multi-objective setting, each edge is given a non-negative cost vector (constant length for each edge in a graph), with each element corresponding to an objective. When these objectives conflict, no single path can optimize all of them simultaneously. MO aims to find a set of Pareto-optimal (non-dominated) solution paths, where a path is Pareto-optimal if no single objective of the path can be improved without causing at least one of the other objectives to deteriorate in quality. During execution, the intermediate path cost vectors form the so-called \textit{Pareto-optimal labels} that comprise potential candidate solutions. However, computing this front is computationally hard \cite{serafini_1987, hansen_1980, BREUGEM_2017, Ehrgott_2005, Papadimitriou_2000}. As the number of objectives and nodes increases, so does the computational complexity and the number of Pareto-optimal solution paths~\cite{Ren2022}. 

Consider an input graph $G=(V,E,c)$ with a set of nodes $V$ and edges $E$. For each edge $e \in E$, there is a non-negative cost vector ${c}(e)$ of length $d$ objectives. Given a source node $v_s$ and a goal node $v_g$ in the graph $G$, the path from $v_s$ to an intermediate node $v$ is defined as a label\footnote{MO literature also identifies a label as a candidate path~\cite{Ren2022, MARTINS198485, sanders_ipdps13}.} $l=(v,\hat{g})$, a tuple represented by a sequence of nodes where each node is connected to its predecessor on the path and $\hat{g}$ denotes the path cost vector from $v_s$ to $v$, calculated as the sum of the cost vectors $c(e)$ for all edges present on the path. For simplicity, we denote $v(l)$ to be the vertex and $\hat{g}(l)$ to be the cost vector contained in $l$.
Since multiple objectives may compete, MO introduces a dominance check such that given two labels $l$ and $l^*$ with $d$ objectives, $l$ dominates $l^*$ (denoted $l \succeq l^*$) if and only if they share the same node ($v(l)=v(l^*)$) and $\hat{g}(l) \succeq \hat{g}(l^*)$ (i.e., $\hat{g}(l)[i] \leq \hat{g}(l^*)[i], \forall i \in {1,2,...,d}$, and $\hat{g}(l)[i] < \hat{g}(l^*)[i], \exists i \in {1,2,...,d}$). 
All non-dominated paths from $v_s$ to $v_g$ constitute the \emph{Pareto-optimal} solution set. MO aims to find a \emph{cost-unique} Pareto-optimal solution set where no two labels in the subset have the same cost vector. 

\begin{figure}
    \centering
    \includegraphics[width=\linewidth]{figs/LabelFigureBig.pdf}
    \caption{Comparison of (a) single objective vs. (b) multi-objective intermediate labels between the source node $v_s$ and an intermediate node $v$. A new candidate label $l^*$ is shown alongside the associated label-level operations.}
    \label{fig:Label}
\end{figure}

In a single-objective shortest path, there can be only one minimum solution label for each node in the graph. 
As seen in Figure~\ref{fig:Label}(a), this is computed using a singular cost comparison for a given node in the graph. However, MO does not guarantee a single solution path, as multiple non-dominated labels can exist from the start node to any other node in the graph.
When a new candidate label $l^*$ is discovered, a \textbf{dominance check} must be performed to verify if the accumulated cost vector for any previous labels for that node $v$ dominates the cost vector for $l^*$. The label $l^*$ must be compared with all previously found non-dominated labels to $v$, as illustrated in Figure~\ref{fig:Label}(b). If it is not dominated, then the mutually non-dominated set of labels at $v$ must be updated with the new label, possibly \textbf{pruning} existing solutions if the new label dominates them. 
These dominance and pruning checks are expensive, especially as the number of non-dominated labels increases with the number of nodes and objectives~\cite{ijcai2023p757}.

To efficiently compute the Pareto-optimal solution front, MO algorithms are developed based on the multiobjective extension of the A* algorithm designed for single-objective search~\cite{10.1145/1754399.1754400, ICAPS20paper199, 10.1145/115234.115368}. Multi-objective A* (MOA*), unlike A* that exits once the first solution is found, needs to store a set of Pareto-optimal labels to the goal node. 
A New Approach to Multi-Objective A* (NAMOA*)~\cite{10.1145/1754399.1754400} and its variants EMOA*, NAMOA*-dr, LTMOA*, and LazyLTMOA* use \textbf{consistent heuristic} for the A* search to ensure that an exact set of solution labels are computed to the goal node. 
%NAMOA* handles an arbitrary number of objectives to compute an exact set of solution paths, thus serving as the basis for the proposed MOS algorithm in this paper.  
A heuristic vector $\hat{h}(v)$ is an \emph{admissible} heuristic {such that it dominates all Pareto-optimal solution labels from node $v$ to the goal node~\cite{10.1145/1754399.1754400}.} One method to derive this heuristic is to compute the single-source shortest path (SSSP) from the goal node for each objective. The resulting shortest paths to all nodes in the graph represent the lower bound path costs, defined as $\hat{h}(v(l))$.
A vector $\hat{f}(l)$ denotes the estimated path cost from the start node to the goal node for a given label, calculated as $\hat{f}(l) = \hat{g}(l) + \hat{h}(v(l))$. This heuristic cost is used in MO algorithms to efficiently compute the non-dominated paths, to find a \emph{cost-unique} Pareto-optimal solution set where no two labels in the subset have the same cost vector.

MO algorithms are a unique class of graph processing as they execute a nondeterministic number of labels for each node in the graph, and each label performs an unstructured and variable number of computations. Depending on the graph characteristics and how the objectives compete, each label per node may explore an arbitrary number of adjacent labels. For each neighbor label, a nondeterministic number of labels must be compared via dominance and pruning checks. The complexity of these operators relies on the number of candidate labels being compared, which increases with the number of nodes and objectives. Given the irregular complexity of each label processed, MO algorithms center around key data structures and operators.

\vspace{0.05cm}
\noindent \textbf{\emph{OPEN}} data structure is a queue of labels prioritized by $\hat{f}(l)$ in increasing lexicographic order of objectives. The labels can be prioritized based on the lexicographic order to guarantee that a Pareto-optimal label is extracted and processed in each iteration (or frontier in traditional graph terminology). A priority queue (PQ) data structure can be used to ensure a label with the highest chance of remaining in the final solution is processed. Other options for organizing labels include FIFO or Stack organization of labels, which introduces complexity and work efficiency tradeoffs.

\vspace{0.05cm}
\noindent \textbf{\emph{Frontier Set}} data structure $\alpha(u)$ for each node $u$ holds all non-dominated labels $l$ at node $u$. Each label in $\alpha(u)$ is a non-dominated partial solution path from $v_s$ to $u$. For illustration purposes, $\alpha(v_s)$ maintains the Pareto-optimal solution from $P$ holding the frontier set at the goal node, $v_s$. The output of the MO algorithm is the exact set of Pareto-optimal solutions in $P$.

\vspace{0.05cm}
\noindent \textbf{\emph{NotDominated}~(l, S, cost)} operator compares the label $l$ with labels in a given set $S$ to verify if a label exists in $S$ that dominates $l$. The cost specifies whether the path-cost ($\Hat{g}$) or heuristic-cost ($\Hat{f}$) is used for the dominance checks.
It returns \emph{false} if $l$ is dominated by a label in $S$, and returns \emph{true} otherwise.

\vspace{0.1cm}
\noindent \textbf{\emph{Prune}~(S, l, cost)} operator searches through all labels in a given set $S$, and removes all labels in $S$ that are dominated by $l$ using path-cost ($\Hat{g}$) or heuristic-cost ($\Hat{f}$).

Next, representative MO algorithms are described to illustrate the execution flow of multi-objective graph problems.

\subsubsection{NAMOA* Algorithm}


\begin{algorithm}[!ht]
\captionsetup{font=small}
\footnotesize
\caption{NAMOA* MO Algorithm original}\label{alg:NAMOA}
\begin{algorithmic}[1]
    \While{OPEN $\neq \emptyset$}
        \State $l \gets$ OPEN.pop()
        \State $G_{OP}(v(l))$.remove($l$)
        \State $G_{CL}(v(l))$.insert($l$)
        \If{$v(l)=v_g$}
            \State Prune($OPEN, l, \Hat{f})$
            \State Prune($P, l, \Hat{f})$ \Comment{skip, if PQ OPEN}
            \If{NotDominated(l, P, $\Hat{f}$)}
                \State $P$.insert($l$)
            \EndIf
        \Else 
            \ForAll {$v'\in$ GetNeighbors($v(l)$)}
                \State $l'\gets (v', \hat{g}(l) + \hat{c}(v(l),v'))$; parent($l'\gets l$)
                \State $\Hat{f}(l')\gets \hat{g}(l') + \hat{h}(v(l'))$
                \If{NotDominated($l', P,\Hat{f}$)) \textbf{and}\\
                \hspace{5.6em}NotDominated($l',G_{OP}(v(l')),\Hat{g}$) \textbf{and}\\
                \hspace{5.6em}NotDominated($l',G_{CL}(v(l')),\Hat{g}$)}
                    \State Prune($G_{CL}(v(l')),l',\Hat{g})$ \Comment{skip, if PQ OPEN}
                    \State Prune($G_{OP}(v(l')),l',\Hat{g})$
                    \State OPEN.insert($l'$)
                    \State $G_{OP}(v(l'))$.insert($l'$)
                \EndIf
            \EndFor
        \EndIf
    \EndWhile
    \State \Return $P$
\end{algorithmic}
\end{algorithm}

The open ($G_{OP}$) is organized as a per-node set and mirrors all labels in OPEN for efficient execution of the NotDominated and Prune operators. The closed ($G_{CL}$) contains the non-dominated solution labels in the frontier set of each node, while $P$ manages the frontier set at the goal node. 

Algorithm~\ref{alg:NAMOA} lists the pseudocode for the execution flow of NAMOA*. Assuming a priority queue OPEN, at each iteration, the label with lexicographically lowest heuristic-cost $\hat{f}$ is extracted (line 2). This label $l$ is removed from $G_{OP}$ and inserted into $G_{CL}$ (lines 3 and 4) to update the frontier set. 

If $l$ is at the goal node ($v(l)=v_g$, line 5), all labels heuristic-dominated by $l$ in OPEN are pruned (line 6). This operation requires a computationally expensive full index search through OPEN. The pruned labels are also removed from $G_{OP}$. To ensure the $P$ set contains only globally Pareto-optimal solution labels, the labels in $P$ dominated by $l$ are pruned (line 7). The sequential PQ-based OPEN guarantees that the label extracted is globally Pareto-optimal and no label already inserted in $P$ is dominated by the current label. However, when OPEN is relaxed, this guarantee is invalid, and thus, the prune $P$ is needed. NotDominated is called to check if $l$ is not dominated by any labels in $P$ (line 8), and if successful, $l$ is inserted into the Pareto-optimal solution front $P$ (line 9). Since $P$ contains only goal node labels, either path-cost ($\Hat{g}$) or heuristic-cost ($\Hat{f}$) can be used for the Prune and NotDominated operators on $P$.


If $l$ is \emph{not} at the goal node, then all neighbors $v'$ of $v(l)$ are explored (line 11). For each $v'$, a new candidate label $l'$ is expanded. The parent pointer of $l'$ is set to $l$ to allow solution path reconstruction once execution concludes (line 12). The new $\hat{f}(l')$ is also computed (line 13). NotDominated is called for $l'$ (line 14) to check if its lower-bound path estimate ($\hat{f}(l')$) is dominated by any goal-node solutions in $P$. The dominance checks are also performed to see if any label path-cost ($\Hat{g}$) in $G_{OP}(v(l'))$ or $G_{CL}(v(l'))$ dominates $l'$ (lines 15-16). If $l'$ is not dominated, all labels from $G_{OP}(v(l'))$ and $G_{CL}(v(l'))$ dominated by $l'$ are pruned (lines 17-18). The labels pruned from $G_{OP}$ are also pruned from OPEN. The label $l'$ is now inserted into OPEN and $G_{OP}$ (lines 19-20). For the same reason as the goal node label, pruning of $G_{CL}$ is unnecessary (line 17) for PQ-based OPEN, as only globally Pareto-optimal labels are inserted for node $v'$. Thus, $l'$ will not find any label to dominate in $G_{CL}(v(l'))$.

Once OPEN is empty, the algorithm terminates, and $P$ contains the final labels representing the Pareto-optimal solution paths (also referred to as exact solutions). 


\subsubsection{NAMOA* Variant}

\begin{algorithm}[!ht]
\captionsetup{font=small}
\footnotesize
\caption{NAMOA* variant MO Algorithm}\label{alg:NAMOA2025}
\begin{algorithmic}[1]
    \While{OPEN $\neq \emptyset$}
        \State $l \gets$ OPEN.pop()
        \State $G_{OP}(v(l))$.remove($l$)
        \If{NotDominated(l, P, $\Hat{f}$)}
            \State $G_{CL}(v(l))$.insert($l$)
            \If{$v(l)=v_g$}
                \State Prune($P, l, \Hat{f})$ \Comment{skip, if PQ OPEN}
                \State $P$.insert($l$)
            \Else 
                \ForAll {$v'\in$ GetNeighbors($v(l)$)}
                    \State $l'\gets (v', \hat{g}(l) + \hat{c}(v(l),v'))$; parent($l'\gets l$)
                    \State $\Hat{f}(l')\gets \hat{g}(l') + \hat{h}(v(l'))$
                    \If{NotDominated($l', P,\Hat{f}$) \textbf{and}\\
                    \hspace{7.1em}NotDominated($l',G_{CL} (v(l')),\Hat{g}$) \textbf{and}\\
                    \hspace{7.1em}NotDominated($l',G_{OP}(v(l')),\Hat{g}$)}
                    \State Prune($G_{CL}(v(l')),l',\Hat{g})$ \Comment{skip, if PQ OPEN}
                    \State Prune($G_{OP}(v(l')),l',\Hat{g})$
                    \State OPEN.insert($l'$)
                    \State $G_{OP}(v(l'))$.insert($l'$)
                    \EndIf
                \EndFor
            \EndIf 
        \EndIf
    \EndWhile
    \State \Return $P$
\end{algorithmic}
\end{algorithm}

%\vspace{1em}

Algorithm~\ref{alg:NAMOA2025} lists the pseudocode for a NAMOA* variant from the literature. 
When a label $l$ is extracted from OPEN and $G_{OP}$, it is pruned if heuristic-cost ($\Hat{f}$) is dominated by another label in $P$ (line 4). Otherwise, the label is inserted into $G_{CL}$ (line 5) and expanded. 
If $l$ is at the goal node, this variant inserts into $P$. The prune OPEN full index search is not needed since the dominance check on label extraction (line 4) lazily prunes labels from OPEN. 
If $l$ is \emph{not} at the goal node, all operators consistent with Algorithm~\ref{alg:NAMOA} are carried out.

\subsubsection{LTMOA* (or EMOA*) Algorithm}

\begin{algorithm}[!ht]
\captionsetup{font=small}
\footnotesize
\caption{LTMOA* Linear Time MO Algorithm}\label{alg:LTMOA}
\begin{algorithmic}[1]
    \While{OPEN $\neq \emptyset$}
        
        \State $l \gets$ OPEN.pop()

        \If{NotDominated($l, P,\Hat{f}$) \textbf{and}\\
        \hspace{2.7em}NotDominated($l,G_{CL} (v(l)),\Hat{g}$)}

            \State Prune($G_{CL}(v(l)),l,\Hat{g})$ \Comment{skip, if PQ OPEN??}
            \State $G_{CL}(v(l))$.insert($l$)

            \If{$v(l)=v_g$}
                \State Prune($P, l, \Hat{f})$ \Comment{skip, if PQ OPEN. EMOA* does not skip!}
                \State $P$.insert($l$)
            \Else 
                %\State Prune($G_{CL}(v(l)),l,\Hat{g})$ 
                %\State $G_{CL}(v(l))$.insert($l$)
                \ForAll {$v'\in$ GetNeighbors($v(l)$)}
                    \State $l'\gets (v', \hat{g}(l) + \hat{c}(v(l),v'))$; parent($l'\gets l$)
                    \State $\Hat{f}(l')\gets \hat{g}(l') + \hat{h}(v(l'))$
                    \If{NotDominated($l', P,\Hat{f}$) \textbf{and}\\
                    \hspace{7.1em}NotDominated($l',G_{CL} (v(l')),\Hat{g}$)}
                        %\State Prune($G_{CL}(v(l')),l',\Hat{g})$ \Comment{skip, if PQ OPEN}
                        %\State Prune($G_{OP}(v(l')),l',\Hat{g})$
                        \State OPEN.insert($l'$)
                        %\State $G_{OP}(v(l'))$.insert($l'$)
                    \EndIf
                \EndFor
            \EndIf 

        \EndIf

    \EndWhile
    \State \Return $P$
\end{algorithmic}
\end{algorithm}

Algorithm~\ref{alg:LTMOA} lists the pseudocode for Linear Time MOA* (LTMOA*) that performs dominance
check in linear time using a linked list or an array to store the $G_{CL}$ for each node. LTMOA* does not maintain the $G_{OP}$. When extracting (label $l$) or generating (label $l'$), it prunes if heuristic-cost ($\Hat{f}$) is dominated by another label in $P$ (lines 3 and 14), and path-cost ($\Hat{g}$) is dominated by another label in $G_{CL}$ for that node (lines 4 and 15). Suppose the extracted or generated label is dominated in either $G_{CL}$ or $P$. In that case, it is discarded, and the current iteration ends since this label cannot lead to a cost-unique Pareto-optimal solution. LTMOA* does not check if the generated label $l'$ is dominated by labels in OPEN associated with the same node. Thus, multiple labels for the same node dominating each other may co-exist in OPEN. Since no pruning is done with labels in OPEN, there is a risk of having a larger OPEN compared to NAMOA*. 

After determining the label $l$ for expansion, it is used to update the frontier set $\alpha(v(l))$. All labels from $G_{CL}(v(l))$ dominated by $l$ are pruned (line 5), and label $l$ is inserted into $G_{CL}$ (line 6).

If $l$ is at the goal node ($v(l)=v_g$, line 7), this means a cost-unique solution is found. First, all labels from $P$ dominated by $l$ are pruned (line 8). This operation may not be required if PQ OPEN extracts globally pareto-optimal labels, ensuring label $l$ does not dominate any labels in $P$. Then, label $l$ is inserted into $P$ (line 9). 

%If $l$ is \emph{not} at the goal node, all labels from $G_{CL}(v(l))$ dominated by $l$ are pruned (line 9), and label $l$ is inserted into $G_{CL}$ (line 10). Then, 
If $l$ is \emph{not} at the goal node, the label $l$ is expanded. For each $v'$, a new candidate label $l'$ is expanded (line 12) and $\hat{f}(l')$ is also computed (line 13). Then, dominance checks for labels in $P$ and $G_{CL}(v(l'))$ are invoked on label $l'$ to verify if $l'$ is dominated and should be discarded or not. If $l'$ is not dominated, it is added to OPEN (line 16). 
%For the same reason as the goal node label, the pruning of $G_{CL}$ is unnecessary (line 16) for PQ-based OPEN since only globally Pareto-optimal labels are inserted in it for the node $v'$. Thus, $l'$ will not find any label to dominate in $G_{CL}(v(l'))$.

When generating a new label $l'$, LTMOA* performs fewer dominance checks compared to NAMOA* as it does not need to operate over $G_{OP}(v(l'))$. NAMOA* only iterates over $P$ when extracting a label, but iterates over $G_{OP}(v(l'))$ twice; first to check if any label path cost ($\hat{g}$) in $G_{OP}(v(l'))$ dominates $l'$, and if $l'$ is not dominated, all labels from $G_{OP}(v(l'))$ dominated by $l'$ are pruned. On the other hand, LTMAO* (and EMOA*) does not perform the dominance checks with $G_{OP}(v(l'))$. Instead, it `lazily' performs more dominance checks over $G_{CL}(v(l))$ and $P$ when it extracts a label $l$ for expansion. Therefore, LTMOA* (or EMOA*) performs fewer dominance checks for a label if the size of $G_{CL}$ for a node is smaller than $G_{OP}$.



\subsubsection{LazyLTMOA* Variant} 

\begin{algorithm}[!ht]
\captionsetup{font=small}
\footnotesize
\caption{LazyLTMOA* Variant of LTMOA*}\label{alg:LazyLTMOA}
\begin{algorithmic}[1]
    \While{OPEN $\neq \emptyset$}
        
        \State $l \gets$ OPEN.pop()

        \If{NotDominated($l, P,\Hat{f}$) \textbf{and}\\
        \hspace{2.7em}NotDominated($l,G_{CL} (v(l)),\Hat{g}$)}

            \State Prune($G_{CL}(v(l)),l,\Hat{g})$ \Comment{skip, if PQ OPEN??}
            \State $G_{CL}(v(l))$.insert($l$)

            \If{$v(l)=v_g$}
                \State Prune($P, l, \Hat{f})$ \Comment{skip, if PQ OPEN. EMOA* does not skip!}
                \State $P$.insert($l$)
            \Else 
                %\State Prune($G_{CL}(v(l)),l,\Hat{g})$ 
                %\State $G_{CL}(v(l))$.insert($l$)
                \ForAll {$v'\in$ GetNeighbors($v(l)$)}
                    \State $l'\gets (v', \hat{g}(l) + \hat{c}(v(l),v'))$; parent($l'\gets l$)
                    \State $\Hat{f}(l')\gets \hat{g}(l') + \hat{h}(v(l'))$
                    \State OPEN.insert($l'$)%\If{NotDominated($l', P,\Hat{f}$) \textbf{and}\\
                    %\hspace{7.1em}NotDominated($l',G_{CL} (v(l')),\Hat{g}$)}
                        %\State Prune($G_{CL}(v(l')),l',\Hat{g})$ \Comment{skip, if PQ OPEN}
                        %\State OPEN.insert($l'$)
                    %\EndIf
                \EndFor
            \EndIf 

        \EndIf

    \EndWhile
    \State \Return $P$
\end{algorithmic}
\end{algorithm}

The LTMOA* is extended using an even `lazier' approach, where dominance checks during label generation are removed. Algorithm~\ref{alg:LazyLTMOA} lists the pseudocode for a LazyLTMOA* that performs dominance checks only when a label $l$ is extracted from OPEN. It only iterates over $G_{CL}$ and $P$ when extracting a label from OPEN, and hence performs the fewest dominance checks.



\section{Characterization}

Runtime breakdown:
\begin{itemize}
    \item OPEN extraction time
    \item Dominance check P/COSTS
    \item Dominance check GOP/GCL
    \item Prune P/COSTS
    \item Prune GOP/GCL
    \item Prune OPEN/GOP
    \item Update time
\end{itemize}
\begin{figure}
    \centering
    \includegraphics[width=1.1\linewidth]{./figs/runtime breakdown.png}
    \caption{Runtime Breakdown in order or Increasing Labels DIMACS NY}
    \label{fig:placeholder}
\end{figure}

\begin{figure}
    \centering
    \includegraphics[width=1.1\linewidth]{./figs/runtime breakdown_NOHOLES.png}
    \caption{Runtime Breakdown in order or Increasing Labels DIMACS NY NO HOLES}
    \label{fig:placeholder}
\end{figure}

Label breakdown:
\begin{itemize}
    \item Extracted
    \item Expanded
    \item Generated
    \item Inserted
\end{itemize}
Label lifecycle distribution:
expanded, (extracted-expanded), (inserted-extracted), (generated-inserted)

\begin{figure}
    \centering
    \includegraphics[width=1.1\linewidth]{./figs/pkl_runtime_breakdown.png}
    \caption{Runtime Breakdown in order or Increasing Labels PKL }
    \label{fig:placeholder}
\end{figure}

\begin{figure}
    \centering
    \includegraphics[width=1.1\linewidth]{./figs/pkl_runtime_breakdownNOHOLES.png}
    \caption{Runtime Breakdown in order or Increasing Labels PKL NO HOLES ALGO}
    \label{fig:placeholder}
\end{figure}

\begin{figure}
    \centering
    \includegraphics[width=0.5\linewidth]{./figs/MAG stage breakdown CPU.png}
    \caption{CPU runtime breakdown per stage}
    \label{fig:placeholder}
\end{figure}

Number of dominance checks




Average label latency: runtime/Labels processed
\begin{table}[htbp]
\caption{Performance Metrics of Different Algorithms (DIMACS)}
\begin{center}
\scriptsize
\begin{tabular}{|c|c|c|c|c|c|}
\hline
\textbf{Algo} & \makecell{\textbf{Dominance}\\\textbf{Checks}} & \makecell{\textbf{Labels}\\\textbf{Processed}} & \textbf{Latency} & \textbf{Throughput} & \makecell{\textbf{Checks/}\\\textbf{Label}} \\
\hline
1 & 19\,970\,796\,949 & 3\,593\,792 & 2.35E-05 & 42\,599.20 & 5\,557 \\
\hline
2 & 30\,458\,761\,118 & 3\,701\,130 & 2.79E-05 & 35\,786.10 & 8\,230 \\
\hline
3 & 33\,471\,120\,546 & 4\,067\,899 & 2.17E-05 & 46\,113.60 & 8\,228 \\
\hline
4 & 19\,544\,490\,792 & 10\,984\,669 & 5.65E-06 & 176\,973.50 & 1\,779 \\
\hline
5 & 17\,316\,325\,383 & 10\,984\,669 & 5.48E-06 & 182\,450.00 & 1\,576 \\
\hline
\end{tabular}
\label{tab:algo_metrics_new}
\end{center}
\end{table}

\begin{table}[htbp]
\caption{Performance Metrics of Different Algorithms (DIMACS NO HOLES)}
\begin{center}
\scriptsize
\begin{tabular}{|c|c|c|c|c|c|}
\hline
\textbf{Algo} & \makecell{\textbf{Dominance}\\\textbf{Checks}} & \makecell{\textbf{Labels}\\\textbf{Processed}} & \textbf{Latency} & \textbf{Throughput} & \makecell{\textbf{Checks/}\\\textbf{Label}} \\
\hline
1 & 19\,972\,477\,998 & 3\,593\,792 & 1.88E-05 & 53\,119.90 & 5\,557 \\
\hline
2 & 30\,460\,402\,820 & 3\,701\,130 & 2.27E-05 & 44\,084.65 & 8\,230 \\
\hline
3 & 33\,471\,065\,550 & 4\,067\,899 & 2.21E-05 & 45\,325.86 & 8\,228 \\
\hline
4 & 19\,544\,432\,630 & 10\,984\,669 & 5.68E-06 & 176\,028.03 & 1\,779 \\
\hline
5 & 17\,316\,325\,383 & 10\,984\,669 & 5.48E-06 & 182\,449.99 & 1\,576 \\
\hline
\end{tabular}
\label{tab:algo_metrics_new}
\end{center}
\end{table}



\begin{table}[htbp]
\caption{Performance Metrics of Different Algorithms (PKL)}
\begin{center}
\scriptsize
\begin{tabular}{|c|c|c|c|c|c|}
\hline
\textbf{Algo} & \makecell{\textbf{Dominance}\\\textbf{Checks}} & \makecell{\textbf{Labels}\\\textbf{Processed}} & \textbf{Latency} & \textbf{Throughput} & \makecell{\textbf{Checks/}\\\textbf{Label}} \\
\hline
1 & 8\,225\,025\,265 & 290\,281 & 2.18E-04 & 4\,583 & 28\,335 \\
\hline
2 & 11\,074\,271\,144 & 433\,341 & 1.17E-04 & 8\,525 & 25\,556 \\
\hline
3 & 13\,025\,580\,242 & 931\,023 & 4.13E-05 & 24\,191 & 13\,991 \\
\hline
4 & 7\,566\,974\,873 & 2\,603\,992 & 9.84E-06 & 101\,626 & 2\,906 \\
\hline
5 & 6\,792\,867\,119 & 2\,603\,992 & 9.38E-06 & 106\,585 & 2\,609 \\
\hline
\end{tabular}
\label{tab:algo_metrics_new}
\end{center}
\end{table}

\begin{table}[htbp]
\caption{Performance Metrics of Different Algorithms (PKL NO HOLES)}
\begin{center}
\scriptsize
\begin{tabular}{|c|c|c|c|c|c|}
\hline
\textbf{Algo} & \makecell{\textbf{Dominance}\\\textbf{Checks}} & \makecell{\textbf{Labels}\\\textbf{Processed}} & \textbf{Latency} & \textbf{Throughput} & \makecell{\textbf{Checks/}\\\textbf{Label}} \\
\hline
1 & 8\,103\,239\,809 & 290\,281 & 1.52E-04 & 6\,574.11 & 27\,915.13 \\
\hline
2 & 10\,801\,312\,904 & 433\,341 & 8.40E-05 & 11\,907.98 & 24\,925.68 \\
\hline
3 & 13\,025\,576\,241 & 931\,023 & 4.16E-05 & 24\,061.38 & 13\,990.61 \\
\hline
4 & 7\,566\,971\,010 & 2\,603\,992 & 9.97E-06 & 100\,275.71 & 2\,905.91 \\
\hline
5 & 6\,792\,867\,119 & 2\,603\,992 & 9.38E-06 & 106\,584.59 & 2\,608.64 \\
\hline
\end{tabular}
\label{tab:algo_metrics_new}
\end{center}
\end{table}

We may want to track the number of labels in GCL at the end 


DIMACS NY road network graphs:
\begin{itemize}
    \item L-noH, L-namoa: 5, 2, 29, 39, 44, 48, \textbf{35}
    \item M-noH, L-namoa: 47, 46, \textbf{50}, 21, 30, 4
    \item M-noH, M-namoa: 31, 16, 41, \textbf{6}, 7
    \item M2-noH, L-namoa: 20, \textbf{15}, \red{1}
    \item M2-noH, M-namoa: 26, \textbf{37}, 18, 49, 24, 22, 19
    \item H-noH, L-namoa: 40, \textbf{17}, 28, 34, 11
    \item H2-noH, L-namoa: 10, 42, \textbf{25}, 32
    \item H2-noH, M-namoa: \textbf{8}, 45, 33, 12, 36, 14, 13
    \item H2-noH, H-namoa: \blue{38}, 3, 9, 43, 27, 23
\end{itemize}

Increasing noH routes: 35, 50, 21, 6, 16, 30, 15, 37, 1, 11, 17, 10, 25, 8, 38


Both runtime and label behavior of five MOA variants were analyzed across different routes and graphs. Each algorithm was evaluated in terms of runtime components with respect to the label operations, and dominance check activity (Figure 4, 5). Across all algorithms, dominance checks account for the largest share of computation time while contribution of other components varies based on the algorithm. Algorithm 1 performed the full pruning operations, establishing the baseline for the overall number of dominance checks. Algorithm 4 performed well across all routes by processing the highest number of labels but maintaining baseline number of dominance checks what resulted in the extremely lightweight per-label dominance checks. Algorithm 5 achieved the best overall performance by further reducing the total number of dominance checks compared to Algorithm 4 (Table 1, 2). Dominance checks per label ratio demonstrates that it is more efficient to process many labels cheaply than to process fewer labels at a higher cost. This highlights that overall efficiency depends not just on the total number of labels, but on how economically each label is handled within the algorithm.


NEXT STEPS: 
\begin{itemize}
\item 1. Fix OPMOS algo 1. Cost array convert to a vector,  G.OP.DEL, G.CL.DEL also to vectors. Currently running out of G.OP.DEL, COST for small lsi routes with N=1
\item 2. Add modifications to all algos. Start with algo 4 (goal node doesn't have a g.CL entry then
whe don't need to do dom g.CL AND PRUNE.GCL on goal nodes: algo 1 (16, 17), Algo 4(4,5))
\item 3. Convert algo 4 sequential to ompos with no separate bags 
\item 4. Additional ideas: Parallel CPU simd execution with vectorization for algo 1. We have 4x128 simd units, thus we can process 8 labels at a time ( 128/2 = 62  . 8 labels 32 each )
\end{itemize}
\subsubsection{Item 2: Algo 1-4 (Modified) with no Gcl operations on goal node: }


\begin{algorithm}
\captionsetup{font=small}
\footnotesize
\caption{Algo 1 move G\_CL outside the neighbors: NAMOA* MO Algorithm}\label{alg:NAMOA_new}
\begin{algorithmic}[1]
    \While{OPEN $\neq \emptyset$}
        \State $l \gets \text{OPEN.pop()}$
        \State $G_{OP}(v(l)).\text{remove}(l)$
        \State $G_{CL}(v(l)).\text{insert}(l)$
        \If{$v(l) = v_g$}
            \State Prune($\text{OPEN}, l, \hat{f}$)
            \State Prune($P, l, \hat{f}$) \Comment{skip, if PQ OPEN}
            \If{NotDominated($l, P, \hat{f}$)}
                \State $P.\text{insert}(l)$
            \EndIf
        \Else \If{NotDominated($l',G_{CL} (v(l')),\Hat{g}$)}
            \State Prune($G_{CL}(v(l')),l',\Hat{g})$ \Comment{skip, if PQ OPEN}
            \EndIf
             \ForAll {$v'\in$ GetNeighbors($v(l)$)}
                   
                    \If{NotDominated($l', P, \hat{f}$) \textbf{and} NotDominated($l', G_{OP}(v(l')), \hat{g}$)}
                    \Comment{skip, if PQ OPEN}
                    \State Prune($G_{OP}(v(l')), l', \hat{g}$)
                    \State OPEN.insert($l'$)
                    \State $G_{OP}(v(l')).\text{insert}(l')$
                \EndIf
            \EndFor
        \EndIf
    \EndWhile
    \State \Return $P$
\end{algorithmic}
\end{algorithm}

\begin{algorithm}
\captionsetup{font=small}
\footnotesize
\caption{ALgo 2: moved GCL operations before neighbor expansion NAMOA* variant MO Algorithm}\label{alg:NAMOA2025_new}
\begin{algorithmic}[1]
    \While{OPEN $\neq \emptyset$}
        \State $l \gets$ OPEN.pop()
        \State $G_{OP}(v(l))$.remove($l$)
        \If{NotDominated(l, P, $\Hat{f}$)}
            \State $G_{CL}(v(l))$.insert($l$)
            \If{$v(l)=v_g$}
                \State Prune($P, l, \Hat{f})$ \Comment{skip, if PQ OPEN}
                \State $P$.insert($l$)
            \Else \If{NotDominated($l',G_{CL} (v(l')),\Hat{g}$)}
            \State Prune($G_{CL}(v(l')),l',\Hat{g})$ \Comment{skip, if PQ OPEN}
            \EndIf
                \ForAll {$v'\in$ GetNeighbors($v(l)$)}
                    \State $l'\gets (v', \hat{g}(l) + \hat{c}(v(l),v'))$; parent($l'\gets l$)
                    \State $\Hat{f}(l')\gets \hat{g}(l') + \hat{h}(v(l'))$
                    \If{NotDominated($l', P,\Hat{f}$) \textbf{and}\\
                   
                    \hspace{7.1em}NotDominated($l',G_{OP}(v(l')),\Hat{g}$)}
                    
                    \State Prune($G_{OP}(v(l')),l',\Hat{g})$
                    \State OPEN.insert($l'$)
                    \State $G_{OP}(v(l'))$.insert($l'$)
                    \EndIf
                \EndFor
            \EndIf 
        \EndIf
    \EndWhile
    \State \Return $P$
\end{algorithmic}
\end{algorithm}

\begin{algorithm}
\captionsetup{font=small}
\footnotesize
\caption{Algo 3: LTMOA* Linear Time MO Algorithm (Modified to skip GCL operations for goal nodes)}\label{alg:LTMOA_Mod_new}
\begin{algorithmic}[1]
    \While{OPEN $\neq \emptyset$}
        \State $l \gets$ OPEN.pop()
        \If{NotDominated($l, P, \Hat{f}$)}
            \If{$v(l) = v_g$}
                \State Prune($P, l, \Hat{f}$) \Comment{Skip if PQ OPEN. EMOA* does not skip!}
                \State $P$.insert($l$)
            \Else
                
                \ForAll {$v'\in$ GetNeighbors($v(l)$)}
                    \State $l'\gets (v', \hat{g}(l) + \hat{c}(v(l),v'))$; parent($l'\gets l$)
                    \State $\Hat{f}(l')\gets \hat{g}(l') + \hat{h}(v(l'))$
                    \If{NotDominated($l', P,\Hat{f}$) \textbf{and}\\
                    \hspace{7.1em}NotDominated($l',G_{CL} (v(l')),\Hat{g}$)}
                        %\State Prune($G_{CL}(v(l')),l',\Hat{g})$ \Comment{skip, if PQ OPEN}
                        %\State Prune($G_{OP}(v(l')),l',\Hat{g})$
                        \State Prune($G_{CL}(v(l)), l, \Hat{g}$) 
                        \State $G_{CL}(v(l))$.insert($l$)
                        \State OPEN.insert($l'$)
                        %\State $G_{OP}(v(l'))$.insert($l'$)
                    \EndIf
                \EndFor
            \EndIf
        \EndIf
    \EndWhile
    \State \Return $P$
\end{algorithmic}
\end{algorithm}

\begin{algorithm}
\captionsetup{font=small}
\footnotesize
\caption{Algo 4: LazyLTMOA* Variant of LTMOA* (Modified to skip GCL operations for goal nodes)}\label{alg:LazyLTMOA_Mod_new}
\begin{algorithmic}[1]
    \While{OPEN $\neq \emptyset$}
        \State $l \gets$ OPEN.pop()
        \If{NotDominated($l, P, \Hat{f}$)}
            \If{$v(l) = v_g$}
                \State Prune($P, l, \Hat{f}$) \Comment{Skip if PQ OPEN. EMOA* does not skip!}
                \State $P$.insert($l$)
            \ElsIf {NotDominated($l, G_{CL}(v(l)), \Hat{g}$)} 
                \State Prune($G_{CL}(v(l)), l, \Hat{g}$) 
                \State $G_{CL}(v(l))$.insert($l$)
                \ForAll{$v' \in$ GetNeighbors($v(l)$)} 
                    \State $l' \gets (v', \hat{g}(l) + \hat{c}(v(l),v'))$; parent($l') \gets l$
                    \State $\Hat{f}(l') \gets \hat{g}(l') + \hat{h}(v(l'))$
                    \State OPEN.insert($l'$)
                \EndFor
            \EndIf
        \EndIf
    \EndWhile
    \State \Return $P$
\end{algorithmic}
\end{algorithm}





\end{document}